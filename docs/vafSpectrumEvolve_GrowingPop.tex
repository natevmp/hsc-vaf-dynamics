\documentclass[pdftex,12pt,a4paper]{scrartcl}
\usepackage[english]{babel}
\usepackage{amsfonts}
\usepackage{amssymb}
\usepackage{physics}
\usepackage{amsmath}
\usepackage{xfrac}
\usepackage{graphicx}
\usepackage{caption}
\usepackage{subcaption}
\usepackage{placeins}
\usepackage{color}
\usepackage{nicefrac}
\newcommand\ddfrac[2]{\frac{\displaystyle #1}{\displaystyle #2}}
\newcommand{\opn}[1]{\operatorname{#1}}
\newcommand{\E}{\operatorname{E}}
\newcommand{\Var}{\operatorname{Var}}
% \renewcommand{\P}{\mathbb{P}}
\renewcommand{\P}[2]{\mathbb{P} \! \left\lbrace #1 \, \mid \, #2 \right\rbrace }

\begin{document}
\subsection{VAF spectrum dynamics}
To obtain the dynamics of the VAF spectrum we first consider the probabilistic dynamics of the size of a single variant in the population. In particular we wish to find a Fokker-Plank equation for the time evolution of the probability density of the variant's size. We first sketch the procedure for obtaining a well-known result for the Moran model, after which we show how to obtain the expression for a stochastically growing population.

\subsubsection{Single variant Moran dynamics}
In the Moran model the discrete probability distribution of the size of a variant is governed by the transition probabilities $\mathbb{P}\! \left\lbrace l \middle\vert k \right\rbrace$ denoting the likelihood of the variant changing from size $k$ to $l$ after the occurrence of a symmetric division event. These take the tridiagonal form:
\begin{equation}
\left\lbrace
\begin{aligned}
\P{k+1}{k} &= \frac{k}{N} \qty( 1- \frac{k}{N} ) = T_{k+1,k} = T_k\\
\P{k-1}{k} &= \frac{k}{N} \qty( 1- \frac{k}{N} ) = T_{k-1,k} = T_k\\
\P{k}{k} &= 1 - (T_{k+1,k} + T_{k-1,k})
\end{aligned}
\right.
\end{equation}
Assuming events occur with exponentially distributed waiting times at rate $\nu$, the probability distribution of the variant's size $P_k$ can be described through the master equation
\begin{equation}
    \frac{1}{\nu} \dv{P_k}{t} = T_{k-1} P_{k-1} +  -2T_k P_k + T_{k+1}P_{k+1}
\end{equation}
For large population sizes this system of $N$ differential equations can be computational taxing to solve. A diffusion approximation can be made on the assumption that $N$ is large (i.e. there are many states $k=0, 1, \dots, N$), which describes $p(\kappa)$ the probability density of the variant size on the continuous domain $\kappa \in [ 0, N ]$. To this end a Fokker-Planck equation is constructed of the form
\begin{equation}
    \partial_t p(\kappa,t) = -\partial_\kappa A(\kappa,t) p(\kappa, t) + \partial_\kappa^2 B(\kappa,t)p(\kappa, t)/2
\end{equation}
The coefficients are given through the infinitesimal propagator $t(\kappa + \Delta \kappa, t+\Delta t \, \vert \, \kappa, t)$:
\begin{equation}
    \begin{aligned}
        A(\kappa, t) &= \lim_{\Delta t \rightarrow 0} \frac{1}{\Delta t} \int \! d(\Delta \kappa) \Delta \kappa \, t(\kappa+\Delta \kappa, t+\Delta t \, \vert \, \kappa, t) \\
        B(\kappa, t) &= \lim_{\Delta t \rightarrow 0} \frac{1}{\Delta t} \int \! d(\Delta \kappa) (\Delta \kappa)^2 \, t(\kappa+\Delta \kappa, t+\Delta t \, \vert \, \kappa, t)
    \end{aligned}
\end{equation}
The integrals are the first and second moments of a displacement $\Delta \kappa$, and can be obtained using the heuristic:
\begin{equation}
    \begin{aligned}
    \ev{\Delta \kappa}_{k, \Delta t} &\approx \ev{\Delta \kappa}_{k, 1/\eta} \Delta t + \vartheta\qty(\Delta t^2) = \sum_{\Delta k} \Delta k \, T_{k+\Delta k,k} \, \eta \, \Delta t + \vartheta\qty(\Delta t^2) \\
    \ev{(\Delta \kappa^2)}_{k, \Delta t} &\approx \ev{(\Delta \kappa^2)}_{k, 1/\eta} \Delta t + \vartheta\qty(\Delta t^2) = \sum_{\Delta k} (\Delta k)^2 \, T_{k+\Delta k,k} \, \eta \, \Delta t + \vartheta\qty(\Delta t^2) 
    \end{aligned}
\end{equation}
Typically a transformation to frequency space is performed -- $ f = \kappa / N$ -- so that we obtain
\begin{align}
    A(f, t) = 0, \quad B(f, t) = \frac{2\nu}{N^2} f(1-f)
\end{align}
and thus
\begin{equation}\label{eq:FPMoran}
    \partial_t p(f,t) = (\nu / N^2) \, \partial_f^2 f(1-f) p(f,t) 
\end{equation}

\subsubsection{Moran dynamics with growth}
We now consider a population in which there are Moran divisions -- a self-renewal simultaneously accompanied by a loss or differentiation -- occurring at rate $\rho N(t)$, as well as additional divisions which increase the population at rate $\gamma N(t)$. For simplicity we assume the population size $N(t)$ to grow deterministically according to the growth rate. The transition probabilies for a single clone are given by
\begin{equation}
\left\lbrace
    \begin{aligned}
        &\P{k+1, t+\Delta t}{k, t} = \frac{k}{N(t)}\qty(1 - \frac{k}{N(t)})\rho N(t)\Delta t + \frac{k}{N(t)}\gamma N(t)\Delta t\\
        &\P{k-1, t+\Delta t}{k, t} = \frac{k}{N(t)}\qty(1 - \frac{k}{N(t)})\rho N(t)\Delta t \\
        &\P{k, t+\Delta t}{k, t} = 1 - \P{k+1, t+\Delta t}{k, t} - \P{k-1, t+\Delta t}{k, t}
    \end{aligned}
\right.
\end{equation}
Using the previously described heuristic the coefficients of the Fokker-Planck equation become
\begin{equation}
    \begin{aligned}
        A(\kappa,t) &= \kappa \gamma \\
        B(\kappa,t) &= 2 \kappa \qty[1 - \kappa/N ]\rho + \kappa \gamma
    \end{aligned}
\end{equation}
And the full equation becomes
\begin{equation}\label{eq:FPSingleGrowth}
    \partial_t p(\kappa,t) = -\partial_\kappa \gamma \kappa \, p(\kappa,t) + \partial_\kappa^2 \qty[\kappa(1-\kappa/N)\rho + \kappa \gamma/2] \, p(\kappa,t).
\end{equation}
Note that if there is no growth, i.e. $\gamma = 0$, this reduces to the standard Moran Fokker-Planck equation upon performing the transformation $f(\kappa) = \kappa/N$.

\subsubsection{VAF spectrum dynamics}
To obtain the dynamics of the spectrum of variants $v(\kappa,t)$ -- i.e. the number of variants per size -- we simply note that for independently evolving mutants the fluxes between states are identical to the above-derived transition probabilities. We however still require a flux of newly arising variants, which is limited to the state $k=1$
and has amplitude
\begin{equation}
    \mathcal{C} = 2\mu (\rho + \gamma + \phi/2)
\end{equation}
where asymmetric divisions occurring at rate $\phi$ only introduce half as many varants into the population. In the continuous domain we introduce this flux as a Dirac delta function, so that the complete expression then takes the form
\begin{equation}\label{eq:FPVafGrowth}
    \begin{aligned}
        \partial_t v(\kappa,t) = -\partial_\kappa \gamma \kappa \, v(\kappa,t) + \partial_\kappa^2 \qty[\kappa(1-\kappa/N)\rho + \kappa \gamma/2] \, v(\kappa,t) &\\
        + 2\mu( \rho + \gamma + \phi/2 )\delta( \kappa-1 ) &
    \end{aligned}
\end{equation}
While a solution to (\ref{eq:FPMoran}) is known, it is rather unwieldy, which does not bode well for the more complex expressions (\ref{eq:FPSingleGrowth}) and (\ref{eq:FPVafGrowth}). We will here use numerical approximations of their solution, though to obtain these we must be cautious about the discontinuity at $\partial_t v(\kappa=1, t)$ introduced by the incoming flux. We applied a method of lines approach with a finite difference discretization in the size coordinate $\kappa$. In order to optimize performance, we opted for a variable stepsize which is smallest near the singularity, with the distances in space given by
\begin{equation}
    \Delta \kappa_i = 1 + (i-1)\cdot \alpha_i
\end{equation}
where $i \in 1, \dots, n-1$, with $n$ the number of discretized points, and
\begin{equation}
    \alpha_i = 2\frac{N-(n-1)}{(n-1)(n-2)}
\end{equation}
In this formalism the delta function in (\ref{eq:FPVafGrowth}) is approximated as a step function with height $2/(\Delta \kappa_1 + \Delta \kappa_2)$.

\end{document}