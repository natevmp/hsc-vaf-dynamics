\documentclass[pdftex,12pt,a4paper]{scrartcl}
\usepackage[english]{babel}
\usepackage{amsfonts}
\usepackage{amssymb}
\usepackage{physics}
\usepackage{amsmath}
\usepackage{xfrac}
\usepackage{graphicx}
\usepackage{caption}
\usepackage{subcaption}
\usepackage{placeins}
\usepackage{color}
\usepackage{nicefrac}
\newcommand\ddfrac[2]{\frac{\displaystyle #1}{\displaystyle #2}}
\newcommand{\opn}[1]{\operatorname{#1}}
\newcommand{\E}{\operatorname{E}}
\newcommand{\Var}{\operatorname{Var}}
\renewcommand{\P}{\mathbb{P}}


\begin{document}

The number of mutations carried by a single cell is given by the sum over all mutations occurring during each of its past divisions:
\begin{equation}
    m = \sum_i^l u_i
\end{equation}
The expected value of $m$ can easily be found from the law of total expectation
\begin{equation}
    \E (m) = \E(l) \E(u) = \lambda \mu
\end{equation}
The variance of $m$ can be written through the law of total variance
\begin{align}
    \Var(m) &= \E(l) \Var(u) + \E(u)^2 \Var(l) \\
    &= \lambda \mu + \mu^2 \Var(l)
\end{align}
where we have used the fact that the $u_i$ are Poisson distributed with $\E(u) = \Var(u) = \mu$. Subtracting the mean from the variance then gives
\begin{align*}
    \Var(m) - \E(m) &= \mu^2 \Var(l) \\
    \Rightarrow \frac{\Var(m) - \E(m)}{\E(m)} &= \mu \frac{\Var(l)}{\E(l)}
\end{align*}
so finally we can find the mutation rate from
\begin{equation}
    \mu = \qty( \frac{\Var(m)}{\E(m)} - 1 ) \frac{\E(l)}{\Var(l)}
\end{equation}
If the number of divisions is Poisson distributed, then $\E(l) = \Var(l)$ so that $\mu = \Var(m)/\E(m) -1$. This is the compound Poisson estimate which we will denote as $\tilde{\mu}$. If we take the $\mu=1.2$ as true, then the CPD estimate $\tilde{\mu}$ gives an indication of how the system dynamics differ from the Poisson model. In particular, if $\tilde{\mu} < \mu$ then we must have $\Var(l) < \E(l)$, which could be an indication of the existence memory in the system, for example in the form of divisions occurring according to a timed cell cycle. Conversely, if $\tilde{\mu} > \mu$, we have $\Var(l) > \E(l)$, which implies that various lineages actually divided far more or far less than what would be appropriate for a memoryless system. Interestingly, this observation need not be at odds with current knowledge of stem cell dynamics, as the possibility of a quiescence state -- which cells enter and exist according to some complex dynamics -- could potentially lead to an increased variance.
\end{document}

